% Options for packages loaded elsewhere
\PassOptionsToPackage{unicode}{hyperref}
\PassOptionsToPackage{hyphens}{url}
%
\documentclass[
]{article}
\usepackage{amsmath,amssymb}
\usepackage{iftex}
\ifPDFTeX
  \usepackage[T1]{fontenc}
  \usepackage[utf8]{inputenc}
  \usepackage{textcomp} % provide euro and other symbols
\else % if luatex or xetex
  \usepackage{unicode-math} % this also loads fontspec
  \defaultfontfeatures{Scale=MatchLowercase}
  \defaultfontfeatures[\rmfamily]{Ligatures=TeX,Scale=1}
\fi
\usepackage{lmodern}
\ifPDFTeX\else
  % xetex/luatex font selection
\fi
% Use upquote if available, for straight quotes in verbatim environments
\IfFileExists{upquote.sty}{\usepackage{upquote}}{}
\IfFileExists{microtype.sty}{% use microtype if available
  \usepackage[]{microtype}
  \UseMicrotypeSet[protrusion]{basicmath} % disable protrusion for tt fonts
}{}
\makeatletter
\@ifundefined{KOMAClassName}{% if non-KOMA class
  \IfFileExists{parskip.sty}{%
    \usepackage{parskip}
  }{% else
    \setlength{\parindent}{0pt}
    \setlength{\parskip}{6pt plus 2pt minus 1pt}}
}{% if KOMA class
  \KOMAoptions{parskip=half}}
\makeatother
\usepackage{xcolor}
\usepackage[margin=1in]{geometry}
\usepackage{color}
\usepackage{fancyvrb}
\newcommand{\VerbBar}{|}
\newcommand{\VERB}{\Verb[commandchars=\\\{\}]}
\DefineVerbatimEnvironment{Highlighting}{Verbatim}{commandchars=\\\{\}}
% Add ',fontsize=\small' for more characters per line
\usepackage{framed}
\definecolor{shadecolor}{RGB}{248,248,248}
\newenvironment{Shaded}{\begin{snugshade}}{\end{snugshade}}
\newcommand{\AlertTok}[1]{\textcolor[rgb]{0.94,0.16,0.16}{#1}}
\newcommand{\AnnotationTok}[1]{\textcolor[rgb]{0.56,0.35,0.01}{\textbf{\textit{#1}}}}
\newcommand{\AttributeTok}[1]{\textcolor[rgb]{0.13,0.29,0.53}{#1}}
\newcommand{\BaseNTok}[1]{\textcolor[rgb]{0.00,0.00,0.81}{#1}}
\newcommand{\BuiltInTok}[1]{#1}
\newcommand{\CharTok}[1]{\textcolor[rgb]{0.31,0.60,0.02}{#1}}
\newcommand{\CommentTok}[1]{\textcolor[rgb]{0.56,0.35,0.01}{\textit{#1}}}
\newcommand{\CommentVarTok}[1]{\textcolor[rgb]{0.56,0.35,0.01}{\textbf{\textit{#1}}}}
\newcommand{\ConstantTok}[1]{\textcolor[rgb]{0.56,0.35,0.01}{#1}}
\newcommand{\ControlFlowTok}[1]{\textcolor[rgb]{0.13,0.29,0.53}{\textbf{#1}}}
\newcommand{\DataTypeTok}[1]{\textcolor[rgb]{0.13,0.29,0.53}{#1}}
\newcommand{\DecValTok}[1]{\textcolor[rgb]{0.00,0.00,0.81}{#1}}
\newcommand{\DocumentationTok}[1]{\textcolor[rgb]{0.56,0.35,0.01}{\textbf{\textit{#1}}}}
\newcommand{\ErrorTok}[1]{\textcolor[rgb]{0.64,0.00,0.00}{\textbf{#1}}}
\newcommand{\ExtensionTok}[1]{#1}
\newcommand{\FloatTok}[1]{\textcolor[rgb]{0.00,0.00,0.81}{#1}}
\newcommand{\FunctionTok}[1]{\textcolor[rgb]{0.13,0.29,0.53}{\textbf{#1}}}
\newcommand{\ImportTok}[1]{#1}
\newcommand{\InformationTok}[1]{\textcolor[rgb]{0.56,0.35,0.01}{\textbf{\textit{#1}}}}
\newcommand{\KeywordTok}[1]{\textcolor[rgb]{0.13,0.29,0.53}{\textbf{#1}}}
\newcommand{\NormalTok}[1]{#1}
\newcommand{\OperatorTok}[1]{\textcolor[rgb]{0.81,0.36,0.00}{\textbf{#1}}}
\newcommand{\OtherTok}[1]{\textcolor[rgb]{0.56,0.35,0.01}{#1}}
\newcommand{\PreprocessorTok}[1]{\textcolor[rgb]{0.56,0.35,0.01}{\textit{#1}}}
\newcommand{\RegionMarkerTok}[1]{#1}
\newcommand{\SpecialCharTok}[1]{\textcolor[rgb]{0.81,0.36,0.00}{\textbf{#1}}}
\newcommand{\SpecialStringTok}[1]{\textcolor[rgb]{0.31,0.60,0.02}{#1}}
\newcommand{\StringTok}[1]{\textcolor[rgb]{0.31,0.60,0.02}{#1}}
\newcommand{\VariableTok}[1]{\textcolor[rgb]{0.00,0.00,0.00}{#1}}
\newcommand{\VerbatimStringTok}[1]{\textcolor[rgb]{0.31,0.60,0.02}{#1}}
\newcommand{\WarningTok}[1]{\textcolor[rgb]{0.56,0.35,0.01}{\textbf{\textit{#1}}}}
\usepackage{graphicx}
\makeatletter
\def\maxwidth{\ifdim\Gin@nat@width>\linewidth\linewidth\else\Gin@nat@width\fi}
\def\maxheight{\ifdim\Gin@nat@height>\textheight\textheight\else\Gin@nat@height\fi}
\makeatother
% Scale images if necessary, so that they will not overflow the page
% margins by default, and it is still possible to overwrite the defaults
% using explicit options in \includegraphics[width, height, ...]{}
\setkeys{Gin}{width=\maxwidth,height=\maxheight,keepaspectratio}
% Set default figure placement to htbp
\makeatletter
\def\fps@figure{htbp}
\makeatother
\setlength{\emergencystretch}{3em} % prevent overfull lines
\providecommand{\tightlist}{%
  \setlength{\itemsep}{0pt}\setlength{\parskip}{0pt}}
\setcounter{secnumdepth}{-\maxdimen} % remove section numbering
\ifLuaTeX
  \usepackage{selnolig}  % disable illegal ligatures
\fi
\usepackage{bookmark}
\IfFileExists{xurl.sty}{\usepackage{xurl}}{} % add URL line breaks if available
\urlstyle{same}
\hypersetup{
  pdftitle={Homework\_DataViz\_Pany},
  pdfauthor={Phanita Yong},
  hidelinks,
  pdfcreator={LaTeX via pandoc}}

\title{Homework\_DataViz\_Pany}
\author{Phanita Yong}
\date{2024-08-09}

\begin{document}
\maketitle

\subsubsection{Homework for Data
Visualization}\label{homework-for-data-visualization}

\subsection{Create 5 charts}\label{create-5-charts}

\begin{Shaded}
\begin{Highlighting}[]
\FunctionTok{library}\NormalTok{(tidyverse)}
\end{Highlighting}
\end{Shaded}

\begin{verbatim}
## -- Attaching core tidyverse packages ------------------------ tidyverse 2.0.0 --
## v dplyr     1.1.4     v readr     2.1.5
## v forcats   1.0.0     v stringr   1.5.1
## v ggplot2   3.5.1     v tibble    3.2.1
## v lubridate 1.9.3     v tidyr     1.3.1
## v purrr     1.0.2     
## -- Conflicts ------------------------------------------ tidyverse_conflicts() --
## x dplyr::filter() masks stats::filter()
## x dplyr::lag()    masks stats::lag()
## i Use the conflicted package (<http://conflicted.r-lib.org/>) to force all conflicts to become errors
\end{verbatim}

\begin{Shaded}
\begin{Highlighting}[]
\FunctionTok{library}\NormalTok{(dplyr)}
\end{Highlighting}
\end{Shaded}

\subsection{1 chart}\label{chart}

\subsection{How does the price of a diamond vary with its carat, cut,
and
color?.}\label{how-does-the-price-of-a-diamond-vary-with-its-carat-cut-and-color.}

\begin{Shaded}
\begin{Highlighting}[]
\FunctionTok{ggplot}\NormalTok{(diamonds,}
       \AttributeTok{mapping =} \FunctionTok{aes}\NormalTok{(carat, price, }\AttributeTok{color =}\NormalTok{ cut)) }\SpecialCharTok{+}
  \FunctionTok{geom\_point}\NormalTok{(}\AttributeTok{size =} \FloatTok{0.8}\NormalTok{, ) }\SpecialCharTok{+}
  \FunctionTok{facet\_wrap}\NormalTok{(}\SpecialCharTok{\textasciitilde{}}\NormalTok{ color) }\SpecialCharTok{+}
  \FunctionTok{labs}\NormalTok{(}\AttributeTok{title =} \StringTok{"How does the price of a diamond vary with its carat, cut, and color?"}\NormalTok{,}
       \AttributeTok{caption =} \StringTok{"Source: ggplot package"}\NormalTok{) }
\end{Highlighting}
\end{Shaded}

\includegraphics{Homework_Dataviz_Pany_files/figure-latex/unnamed-chunk-2-1.pdf}

\subsection{2nd Chart}\label{nd-chart}

\subsection{Density of carat.}\label{density-of-carat.}

\begin{Shaded}
\begin{Highlighting}[]
\FunctionTok{library}\NormalTok{(tidyverse)}
\FunctionTok{library}\NormalTok{(dplyr)}

\NormalTok{base }\OtherTok{\textless{}{-}}\NormalTok{ diamonds }\SpecialCharTok{\%\textgreater{}\%}
  \FunctionTok{select}\NormalTok{(carat, color)}

\FunctionTok{ggplot}\NormalTok{(base,}
       \FunctionTok{aes}\NormalTok{(carat)) }\SpecialCharTok{+}
  \FunctionTok{geom\_density}\NormalTok{(}\AttributeTok{color =} \StringTok{"red"}\NormalTok{) }\SpecialCharTok{+}
  \FunctionTok{facet\_wrap}\NormalTok{(}\SpecialCharTok{\textasciitilde{}}\NormalTok{ color) }\SpecialCharTok{+}
  \FunctionTok{labs}\NormalTok{(}\AttributeTok{title =} \StringTok{"Density of different color in Carat"}\NormalTok{,}
       \AttributeTok{caption =} \StringTok{"Source: ggplot package"}\NormalTok{)}
\end{Highlighting}
\end{Shaded}

\includegraphics{Homework_Dataviz_Pany_files/figure-latex/unnamed-chunk-3-1.pdf}

\subsection{3.How price and cut vary?}\label{how-price-and-cut-vary}

\begin{Shaded}
\begin{Highlighting}[]
\FunctionTok{ggplot}\NormalTok{(diamonds,}
        \FunctionTok{aes}\NormalTok{(cut, price, }\AttributeTok{color =}\NormalTok{ cut)) }\SpecialCharTok{+}
        \FunctionTok{geom\_boxplot}\NormalTok{()}\SpecialCharTok{+}
        \FunctionTok{theme\_minimal}\NormalTok{() }\SpecialCharTok{+}
        \FunctionTok{labs}\NormalTok{(}\AttributeTok{title =} \StringTok{"Carat vs. Prices"}\NormalTok{,}
             \AttributeTok{caption =} \StringTok{"Source: ggplot package"}\NormalTok{)}
\end{Highlighting}
\end{Shaded}

\includegraphics{Homework_Dataviz_Pany_files/figure-latex/unnamed-chunk-4-1.pdf}
\#\# 4.How price and carat vary across different clarity levels

\begin{Shaded}
\begin{Highlighting}[]
\FunctionTok{ggplot}\NormalTok{(}\AttributeTok{data =}\NormalTok{ diamonds,}
       \AttributeTok{mapping =} \FunctionTok{aes}\NormalTok{(carat, price, }\AttributeTok{color =}\NormalTok{ clarity))}\SpecialCharTok{+} 
       \FunctionTok{geom\_point}\NormalTok{() }\SpecialCharTok{+}
       \FunctionTok{facet\_wrap}\NormalTok{(}\SpecialCharTok{\textasciitilde{}}\NormalTok{clarity) }\SpecialCharTok{+}
       \FunctionTok{labs}\NormalTok{(}\AttributeTok{title =} \StringTok{"Carat vs. Price by Clarity "}\NormalTok{,}
            \AttributeTok{caption =} \StringTok{"Source: ggplot package"}\NormalTok{)}
\end{Highlighting}
\end{Shaded}

\includegraphics{Homework_Dataviz_Pany_files/figure-latex/unnamed-chunk-5-1.pdf}

\subsection{5. clarity vs price}\label{clarity-vs-price}

\begin{Shaded}
\begin{Highlighting}[]
\FunctionTok{set.seed}\NormalTok{(}\DecValTok{42}\NormalTok{)}
\NormalTok{df }\OtherTok{\textless{}{-}}\NormalTok{ diamonds }\SpecialCharTok{\%\textgreater{}\%}
  \FunctionTok{sample\_frac}\NormalTok{(}\FloatTok{0.2}\NormalTok{)}

\FunctionTok{ggplot}\NormalTok{(}\AttributeTok{data =}\NormalTok{ df, }
       \FunctionTok{aes}\NormalTok{(clarity, price, }\AttributeTok{color =} \StringTok{\textquotesingle{}salmon\textquotesingle{}}\NormalTok{ ))}\SpecialCharTok{+} 
       \FunctionTok{geom\_point}\NormalTok{() }\SpecialCharTok{+}
       \FunctionTok{labs}\NormalTok{(}\AttributeTok{title =} \StringTok{"Clarity vs. Price"}\NormalTok{,}
            \AttributeTok{caption =} \StringTok{"Source: ggplot package"}\NormalTok{)  }
\end{Highlighting}
\end{Shaded}

\includegraphics{Homework_Dataviz_Pany_files/figure-latex/unnamed-chunk-6-1.pdf}

\end{document}
